\documentclass[12pt]{article}
\usepackage[utf8x]{inputenc}
\usepackage[T2A]{fontenc}
\usepackage[russian]{babel}
\usepackage{amsmath}
\usepackage{amssymb}
\usepackage{wasysym}
\usepackage{gensymb}
\usepackage[left=1cm,right=1cm,top=1cm,bottom=2cm]{geometry}
\usepackage{wrapfig}
\usepackage{graphicx}
\graphicspath{ {./images/} }
\usepackage{hyperref}
\frenchspacing
\pretolerance1000
\newcounter{step}
\setcounter{step}{1}
\newcommand{\reset}{\setcounter{step}{1}}
\newcommand{\step}{(\arabic{step}) \addtocounter{step}{1}}
\newcommand{\bref}[1]{(\ref{#1})}
\newcommand{\R}{\mathbb{R}}
\newcommand{\N}{\mathbb{N}}
\newcommand{\mC}{\mathbb{C}}
\newcommand{\p}{\partial}
\newcommand{\comment}[1]{}
\newcommand{\np}{\newpage\noindent}
\newcommand{\m}{^\text{m}}
\DeclareMathOperator{\const}{const}
\DeclareMathOperator{\Ree}{Re}

\title{\textbf{Полезные формулы}}
\date{2020\\ Декабрь}
\author{Майнор ``Астрофизика'', 2 курс}
\begin{document}
\maketitle
\section{Формулы}
Единицы измерения неочевидных величин приведены в квадратных скобках в системе СИ. Будте осторожны, в СГС могут быть другие константы или они могут вообще отсутствовать. В случае, если формула приведена в СГС, это обязательно будет указано сбоку.
\subsection{Разрешающая способность}
\[\theta=1.22\frac\lambda d\sim\frac\lambda d\]
$\theta$ --- угол в радианах, $\lambda$ --- длина волны, $d$ --- диаметр линзы (зеркала)
\subsection{Связь светового потока и светимости}
\[f=\frac{L}{4\pi d^2}\]
$f\,[\text{Вт}/\text{м}^2]$ --- поток, $L\,[\text{Вт}]$ --- светимость, $d$ --- расстояние до звезды
\subsection{Связь светового потока и видимой звездной величины}
\[\frac{f_1}{f_2}=100^{\frac{-(m_1-m_2)}{5}}\quad\quad m_1-m_2=-2.5\log_{10}\left(\frac{f_1}{f_2}\right)\]
$f\,[\text{Вт}/\text{м}^2]$ --- потоки, $m\,[^\text{m}]$ --- видимые звездные величины
\subsection{Третий закон Кеплера}
\[\left(\frac{a_1}{a_2}\right)^3=\left(\frac{T_1}{T_2}\right)^2\]
$a$ --- большие полуоси орбит, $T$ --- периоды
\subsection{Связь большой полуоси и периода для задачи двух тел}
\[a^3=\frac{G\mu T^2}{4\pi^2}\quad\quad T=\frac{2\pi a^{\frac32}}{\sqrt{G\mu}}\]
$a$ --- большая полуось, $\mu=m_1+m_2$ --- приведенная масса, $T$ --- период
\subsection{Теорема Вириала}
\[E_\text{полн}=-\frac12|E_\text{пот}|\]
\[E_\text{кин}=\frac12|E_\text{пот}|\quad(\text{частный случай})\]
\subsection{Время транзита}
\[\frac{t}{T}=\frac{1}{\pi}\arcsin\frac Ra=\frac1\pi\arcsin R\left(\frac{4\pi^2}{GMT^2}\right)^{\frac13}\]
$t$ --- время транзита, $T$ --- период, $R$ --- радиус звезды, $a$ --- большая полуось, $M$ --- масса звезды
\subsection{Плотность энергии магнитного поля}
\[\varepsilon=\frac{B^2}{8\pi}\quad(\text{СГС})\]
$B\,[\text{Гс}]$ --- магнитная индукция, $\varepsilon\,[\text{Дж}/\text{м}^3]$ --- плотность энергии
\subsection{Эддингтоновская светимость}
\[L_\text{э}=10^{38}\frac{M}{M_{\astrosun}}\,\biggl[\frac{\text{эрг}}{\text{с}}\biggr]\quad(\text{СГС})\]
\subsection{Шварцшильдский радиус}
\[R_\text{ч.д.}=\frac{2GM}{c^2}\]
\subsection{Уравнение Фридмана}
\[H^2=\left(\frac{\dot{R}}{R}\right)=\frac{8\pi G}{3c^2}\varepsilon+c^2\frac{\Lambda}{3}-kc^2\frac{1}{R^2}\]
$H\,[\text{с}^{-1}]$ --- ``постоянная Хаббла'', $R$ --- ``радиус'' Вселенной, $\varepsilon\,[\text{Дж}/\text{м}^3]$ --- плотность энергии вещества,\\
$\Lambda$ --- космологическая постоянная, $k$ --- кривизна Вселенной
\[k=\begin{cases}
1, & \text{замкнутая (шар, Пуанкаре)}\\
0, & \text{плоская (плоскость, Евклид)}\\
-1, & \text{открытая (гиперболоид, Лобачевский)}
\end{cases}\]
\section{Константы и соотношения}
\subsection{Звездные величины}
\begin{center}\begin{tabular}{|c|c|}
\hlineСветило & Видимая зв. величина\\
\hlineВега & 0.03\\
\hlineСириус & -1.46\\
\hlineЛуна & -12.7\\
\hlineСолнце & -26.7\\
\hline
\end{tabular}\end{center}
Абсолютная звездная величина --- это видимая на расстоянии 10 пк.
\subsection{Единицы измерения}
\begin{align*}
1\,\text{эВ}&=1.6\cdot10^{-12}\,\text{эрг}=1.6\cdot10^{-19}\,\text{Дж}\\
1\,\text{а.\,е.}&=1.5\cdot10^{13}\,\text{см}=1.5\cdot10^{11}\,\text{м}\\
1\,\text{св. год}&=9.5\cdot10^{17}\,\text{см}=63\,000\,\text{а.\,е.}\\
1\,\text{пк}&=3.1\cdot10^{18}\,\text{см}=206\,000\,\text{а.\,е.}=3.26\ \text{св. года}
\end{align*}
\subsection{Константы}
\begin{center}\begin{tabular}{|c|c|c|}
\hlineПостоянная & Обозначение & Значение в СИ\\
\hlineГравитационная & G & $6.67\cdot10^{-11}$ м$^3$с$^{-2}$кг$^{-1}$\\
\hlineСкорость света & c & 300\,000\,000 м/с\\
\hlineХаббла & H & $2.2\cdot10^{-18}$ с$^{-1}$\\
\hline
\end{tabular}\end{center}
\end{document}